
\noindent 8. Pode-se generalizar o problema da árvore de busca binária ótima para que inclua na sua descrição uma estimativa do número de buscas mal-sucedidas. Mais concretamente, nessa variante do problema, seriam dados não apenas os números de acessos $a_i$ de cada chave $i$, mas também números $b_i$, para $i = 0, ..., n$, representando uma estimativa do número de buscas mal-sucedidas a elementos entre $i$ e $i+1$ (considere 0 como $-\infty$ e $n-1$ como $+\infty$ para que $b_0$ e $b_n$ sejam o que se espera).
Encontre a recorrência para o custo de uma árvore de busca binária ótima para esta variante do problema. Escreva o algoritmo de programação dinâmica gerado a partir dessa recorrência. Quanto tempo consome o seu algoritmo?\\[6pt]
\textbf{Resposta:} Para generalizarmos o problema da ABB ótima, tal que possamos incluir estimativas de acesso mal sucedidas, vamos pensar em uma ABB que armazena as chaves $k = \langle k_1, k_2, ..., k_n \rangle$ ordenadas e, para cada chave, tenhamos a estimativa do número de acessos a cada elemento de $k$ dado por $a_i$, para todo $i = 1, 2, ..., n$. Essa representação é a que nós já vimos em sala de aula e pode ser vista na figura \ref{fig:6.8-1}.\\

% Set the overall layout of the tree
\tikzstyle{level 1}=[level distance=1.5cm, sibling distance=6cm]
\tikzstyle{level 2}=[level distance=1.5cm, sibling distance=3cm]
\tikzstyle{level 3}=[level distance=1.5cm, sibling distance=2cm]

% Define styles for bags and leafs
\tikzstyle{bag} = [shape=ellipse, draw, align=center,
                   top color=white, bottom color=gray!20]
\tikzstyle{end} = [shape=rectangle, rounded corners, draw, align=center, fill=gray!40]

\begin{figure}[!h]
\centering
\begin{tikzpicture}[grow=down]
\node[bag, label={$a_2$}] {$k_2$}
    child {
        node[bag, label={$a_1$}] {$k_1$}
            edge from parent 
            node[above] {}
    }
    child {
        node[bag, label={$a_4$}] {$k_4$}
            child {
                node[bag, label={$a_3$}] {$k_3$}
                edge from parent 
                node[above] {}
			}
            child {
                node[bag, label={$a_5$}] {$k_5$}
                edge from parent 
                node[above] {}
			}
    };
\end{tikzpicture}
\captionof{figure}{ABB com número de acessos a cada chave}
\label{fig:6.8-1}
\end{figure}

Como temos buscas que são mal sucedidas, ou seja, valores que não estão em $k$, nós devemos estender a ABB de forma a incluir $n+1$ chaves que representam todas essas buscas que, neste caso, denominamos por $q_0, q_1, ..., q_n$. A figura \ref{fig:6.8-2} mostra essa nova representação. Note que, cada nó terminal $k_i$ na estrutura anterior agora recebe novos nós (folhas) que representam as buscas mal sucedidas entre $k_i$ e $k_{i+1}$. Vale ressaltar que, $q_0$ representa todas os valores menores que $k_1$ e $q_n$ representa todos os valores maiores que $k_n$.

Da mesma forma que nas buscas bem sucedidas, temos a estimativa do número de acessos associado a cada chave $q_i$ não encontrada em $k$ que é dado por $b_0, b_1, ..., b_n$.  Por exemplo, se tivéssemos $k = \langle 3, 5, 7, 11, 12\rangle$, se em uma dada ABB $k_3 = 7$ e $k_4 = 11$, o nó terminal $q_3$ representaria os valores não encontrados $\{8, 9, 10\}$.

\begin{figure}[!h]
\centering
\begin{tikzpicture}[grow=down]
\node[bag, label={$a_2$}] {$k_2$}
    child {
        node[bag, label={$a_1$}] {$k_1$}
            child {
                node[end, label={$b_0$}]{$q_0$}
                edge from parent
                node[above] {}
            }
            child {
                node[end, label={$b_1$}]{$q_1$}
                edge from parent
                node[above] {}
            }
            edge from parent 
            node[above] {}
    }
    child {
        node[bag, label={$a_4$}] {$k_4$}
            child {
                node[bag, label={$a_3$}] {$k_3$}
                	child {
                        node[end, label={$b_2$}]{$q_2$}
                        edge from parent
                        node[above] {}
                    }
                	child {
                        node[end, label={$b_3$}]{$q_3$}
                        edge from parent
                        node[above] {}
                    }
                edge from parent 
                node[above] {}
			}
            child {
                node[bag, label={$a_5$}] {$k_5$}
                	child {
                        node[end, label={$b_4$}]{$q_4$}
                        edge from parent
                        node[above] {}
                    }
                	child {
                        node[end, label={$b_5$}]{$q_5$}
                        edge from parent
                        node[above] {}
                    }
                edge from parent 
                node[above] {}
			}
    };
\end{tikzpicture}
\captionof{figure}{ABB estendida adicionando número de buscas mal-sucedidas}
\label{fig:6.8-2}
\end{figure}

Visto isso, podemos definir a \textbf{subestrutura ótima} do problema: se uma ABB é ótima, então as subárvores esquerda e direita também são ótimas. Vale ressaltar que, quando escolhemos uma das chaves, digamos $k_r$ ($i \leq r \leq j$), como a raiz de uma subárvore $k_i, ...,k_j$, a subárvore à esquerda de $k_r$ deve conter as chaves $k_i, ..., k_{r-1}$, bem como as representantes de buscas mal sucedidas $q_{i-1}, ...,q_{r-1}$ e a subárvore à direita de $k_r$ deve conter as chaves $k_{r+1}, ..., k_{j}$ e $q_r, ..., q_j$.

Agora, devemos definir uma solução recursiva para encontrar uma ABB ótima para as chaves $k_1, k_2, ..., k_n$ cujo número esperado de comparações para uma busca bem sucedida $a_1, a_2, ..., a_n$ e o número esperado de comparações para uma busca mal sucedida $b_0, b_1, b_2, ..., b_n$ seja mínimo.

Seja $e[i, j]$ o custo esperado para uma busca em uma ABB ótima com as chaves $i, i+1, ..., j$. Vale a seguinte recorrência para $e[i, j]$:

\begin{align*}
e[i, j] = \left\{\begin{array}{rl}
                    b_{i - 1}, &\mbox{se $i > j$} \\
                    \smash{\displaystyle\min_{i \leq r \leq j}} {e[i, r-1] + e[r+1, j]} + w[i, j], &\mbox{se $i \leq j $}
                \end{array} \right.
\end{align*}

Onde:
\begin{align*}
w[i, j] = \left\{\begin{array}{rl}
                    b_{i - 1}, &\mbox{se $i > j$} \\
                    w[i, j-1] + a_j + b_j, &\mbox{se $i \leq j $}
                \end{array} \right.
\end{align*}

Ou seja, $w[i, j]$ é a soma das estimativas do número de buscas bem e mal sucedidas para um dado $i, j$:
\begin{align*}
w[i, j] = \sum_{l = i}^j a_l + \sum_{l = i-1}^j b_l
\end{align*}

Agora, basta efetuarmos as alterações necessárias no algoritmo dado em sala de aula para calcularmos as matrizes $w$ e $e$. Também vamos armazenar o valor de $r$ em uma matriz $root$, ou seja, qual a raiz foi escolhida para uma determinada subárvore $k_i, ..., k_j$. O algoritmo $\proc{Extended-Optimal-BST}$ mostra o resultado.

\begin{codebox}
\Procname{$\proc{Extended-Optimal-BST}(a, b, n)$}
\li \For $i \gets 1$ \To $n + 1$
\li \Do
        $e[i, i-1] \gets b_{i-1}$
\li     $w[i, i-1] \gets b_{i-1}$
    \End
\li \For $l \gets 1$ \To $n$
\li \Do
        \For $i \gets 1$ \To $n - l + 1$
\li     \Do
            $j \gets i + l - 1$
\li         $w[i, j] \gets w[i, j-1] + a_j + b_j$
\li         $e[i, j] \gets \infty$
\li         \For $r \gets i$ \To $j$
\li         \Do
                $aux \gets e[i, r-1] + e[r+1, j] + w[i, j]$
\li             \If $aux < e[i, j]$
\li             \Then
                    $e[i, j] \gets aux$
\li                 $root[i, j] \gets r$
                \End
            \End
        \End 
    \End
\li \Return $e, root$
\end{codebox}

O custo mínimo da ABB ótima é retornado na posição $e[1, n]$. A tabela \ref{tbl:6-8-1} mostra o tempo de execução do algoritmo $\proc{Extended-Optimal-BST}$. As linhas 4-13 têm 3 \textit{loops} encadeados, que tomam no máximo $n$ iterações cada um, o que nos dá um consumo total $\Theta(n^3)$.

\begin{table}[H]
\centering
\begin{tabular}{|l|l|}
\hline
Linha                   & Tempo \\ \hline
1-3 & $\Theta(n)$ \\ \hline
4-13 & $\Theta(n^3)$ \\ \hline
14 & $\Theta(1)$ \\ \hline
Total & $\Theta(n) + \Theta(n^3) = \Theta(n^3)$ \\ \hline
\end{tabular}
\caption{Tempo de execução}
\label{tbl:6-8-1}
\end{table}