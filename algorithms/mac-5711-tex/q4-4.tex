\noindent 4. (\textbf{CLRS 9.3-5}) Para esta questão, vamos dizer que a mediana de um vetor $A[p..r]$ com números inteiros é o valor que ficaria na posição $A[\floor{(p + r)/2}]$ depois que o vetor $A[p..r]$ fosse ordenado.\\
Dado um algoritmo linear “caixa-preta” que devolve a mediana de um vetor, descreva um algoritmo simples, linear, que, dado um vetor $A[p..r]$ de inteiros distintos e um inteiro $k$, devolve o \textit{k-ésimo} mínimo do vetor. (O \textit{k-ésimo} mínimo de um vetor de inteiros distintos é o elemento que estaria na \textit{k-ésima} posição do vetor se ele fosse ordenado.)\\[6pt]
Resposta\\