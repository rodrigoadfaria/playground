
\noindent 2. \textbf{(CLRS 8.1-1)} Qual a menor profundidade (= menor nível) que uma folha pode ter em uma árvore de decisão que descreve um algoritmo de ordenação baseado em comparações?\\[6pt]
A menor profundidade da árvore de decisão é, coincidentemente, a cota inferior das alturas de todas as árvores de decisão nas quais aparecem uma folha (uma das $n!$ permutações da entrada).

Também podemos dizer que é o melhor caso em tempo de execução de qualquer algoritmo de ordenação baseado em comparações.

Logo, é o caso em que apenas $n - 1$ comparações são realizadas para ordenar o vetor que ocorre, por exemplo, quando o vetor já está ordenado.\\[12pt]