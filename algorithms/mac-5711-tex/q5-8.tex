
\noindent 8. (\textbf{CLRS 8.2-4}) Descreva um algoritmo que, dados $n$ inteiros 
no intervalo de $1$ a $k$, preprocesse sua entrada e então responda em $O(1)$ 
qualquer consulta sobre quantos dos $n$ inteiros dados caem em um intervalo $[a 
\cdots b]$. O preprocessamento efetuado pelo seu algoritmo deve consomir tempo 
$O(n + k)$.
\\[6pt]
\textbf{Resposta}: O algoritmo desenvolvido é baseado no algoritmo de 
\textit{Counting Sort} descrito no livro do Cormen. O algoritmo possuí duas 
rotinas, uma de preprocessamento que devolve um vetor de contagem C e um de 
consulta que retorna a quantidade de elementos conforme o enunciado. O algoritmo 
de preprocessamento abaixo recebe um vetor $A$ de $n$ inteiros qualquer tal que 
seus elementos estão dentro do intervalo $[1 \cdots k]$ e  realiza o 
preprocessamento devolvendo o vetor "$C$" do \textit{counting sort} antes de 
ordena-lo no vetor $B$. Esse vetor possuí o índice do último valor no vetor 
ordenado de cada índice, por exemplo, $C[x] = y$, indica que o último elemento 
$x$ está no índice $y$ do vetor ordenado.

\begin{codebox}
\Procname{$\proc{PreProcessamento}(A, k, n)$}
\li    \For $i \gets 1$ \To $k$
\li    \Do
            $C[i] \gets 0$
       \End
\li    \For $i \gets 1$ \To $n$
\li    \Do
            $C[A[i]] \gets C[A[i]] + 1$
       \End
\li    \For $i \gets 2$ \To $k$
\li    \Do
            $C[i] \gets C[i] + C[i-1]$
       \End
\li    \Return $C$
\End
\end{codebox}
O procedimento $\proc{ContaNoIntervalo}$ recebe o vetor gerado pelo 
$\proc{PreProcessamento}$ e os extremos do intervalo $[a \cdots b]$ e devolve a 
quantidade de elementos do vetor original $A$ que estão dentro desse vetor.
\begin{codebox}
\Procname{$\proc{ContaNoIntervalo}(C, a, b)$}
    \li \If $a \leq 1$
    \li     \Then
            $a' \gets 1$
    \li \Else
            $a' \gets C[a-1] + 1$
        \End
    \li \Return $C[b] - a' + 1$
\End
\end{codebox}
\textbf{Corretude}
\\[6pt]
O procedimento de preprocessamento utiliza o \textit{counting sort} disso 
sabemos que o vetor $C[i]$ contém a quantidade de elementos menores ou iguais a 
$i$ do vetor original, isso implica que o valor de $C[i]$ é a posição do último 
elemento $i$ (não o elemento da posição $i$) do vetor original no vetor 
ordenado, isso implica também que o valor de $C[i-1] + 1$ é a posição do 
primeiro elemento $i$ do vetor original $A$ no vetor ordenado. Disso temos que 
$C[b] - C[a-1] + 1$ é quantidade de elementos com os valores entre $[a \cdots 
b]$ do vetor original. Note que o $if$ do procedimento 
$\proc{ContaNoIntervalo}$, apenas garante que $a-1 \geq 1$ para que não seja 
acessado uma posição fora do vetor $C$, caso isso ocorra consideramos que a 
posição inicial é $1$, note que isso é válido para quando existem valores $1$ em 
$A$ e quando não existem.
\\[6pt]
\textbf{Desempenho}
\\[6pt]
Veja a tabela de gastos em cada linha do procedimento $\proc{PreProcessamento}$:

\begin{table}[H]
\centering
\begin{tabular}{|l|l|}
\hline
Linha                   & Tempo \\ \hline
1 & $\Theta(k)$ \\ \hline
2 & $\Theta(k)$ \\ \hline
3 & $\Theta(n)$ \\ \hline
4 & $\Theta(n)$ \\ \hline
5 & $\Theta(k)$ \\ \hline
6 & $\Theta(k)$ \\ \hline
7 & $\Theta(1)$ \\ \hline
Total & $\Theta(n + k)$ \\ \hline
\end{tabular}
\end{table}
\noindent O procedimento gasta o tempo total $\Theta(n + k)$. O tempo gasto no 
procedimento $\proc{ContaNoIntervalo}$ está expresso na tabela abaixo:
\begin{table}[H]
\centering
\begin{tabular}{|l|l|}
\hline
Linha                   & Tempo \\ \hline
1 & $\Theta(1)$ \\ \hline
2 & $\Theta(1)$ \\ \hline
3 & $\Theta(1)$ \\ \hline
4 & $\Theta(1)$ \\ \hline
Total & $\Theta(1)$ \\ \hline
\end{tabular}
\end{table}
\noindent Portanto o algoritmo gasta tempo $\Theta(n + k)$ para o 
preprocessamento e tempo $\Theta(1)$ para a chamada a contagem dos elementos 
conforme enunciando do exercício.