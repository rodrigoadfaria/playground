
\noindent \textbf{14. Problema 14-4 do CLRS} (Planejando uma festa da empresa) 
O professor Stewart presta consultoria ao presidente de uma corporação que 
está planejando uma festa da empresa. A empresa tem uma estrutura hierárquica; 
isto é, a relação de supervisores forma uma árvore com raiz no presidente. O 
pessoal do escritório classificou cada funcionário com uma avaliação de 
sociabilidade, que é um número real. Para tornar a festa divertida para todos 
os partocipantes, o presidente não deseja que um funcionário e seu supervisor 
imediato participem.
\\[6pt]
\noindent O professor Stewart recebe uma árvore que descreve a estrutura 
hierárquica da corporação, usando a representação de filho da esquerda, irmão 
da direita, usanda para armazenamento de árvores enraizadas (olhe na seção 
10.4 se precisar). Cada nó da árvore contém além dos ponteiros, o nome de um 
funcionário e a ordem de sociabilidade desse funcionário. Escreva um algoritmo 
para compor a lista de convidados que maximize a soma das avaliações de 
sociabilidade dos convidados. Analise o tempo de execução do seu algoritmo.
\\[6pt]
\noindent \textbf{Resposta: } Vamos resolver esse problema com a técnica de 
programação dinâmica.
\\[6pt]
\noindent \textbf{Sub-estrutura ótima}: Seja $T$ a árvore hierarquica da 
empresa, sabemos que dado a solução ótima $S$ que maximiza a soma de 
sociabilidade da empresa, se a raiz de $T$ pertece a essa solução $S$ a 
sociabilidade dessa solução é a soma da sociabilidade do presidente (raiz da 
árvore) e das soluções ótimas das subárvores enraizadas em seus subordinados 
diretamente, note que seus subordinados não podem ser convidados. Se o 
presidente não pertence a solução ótima desse problema então a sociabilidade da 
solução ótima é a soma das sociabilidades das soluções ótimas das subárvores 
enraizadas em seus subordinados direitos. \textbf{Prova:} Vamos assumir por 
contradição que na solução ótima, dado as subárvores enraizadas nos 
subordinados diretos do presidente exista uma delas que não seja ótima, então 
com certeza podemos trocar essa subárvore que não é ótima pela subárvore ótima 
e conseguir uma solução com sociabilidade maior que a solução ótima assumida 
que é uma contradição! Já que a solução ótima maximiza a sociabilidade.
\\[6pt]
\noindent \textbf{Recorrência}: Vamos definir identificar cada nó de nossa 
árvore como o nome do funcionário que o nó representa. Então definimos $c[x]$ 
como:

\begin{framed}

\noindent $c[x] := $  é a soma maxíma da sociabilidade dos convidados da festa 
da árvore enraizada em $x$ tal que dado que um nó foi convidado seu pai não foi.

\end{framed}

\noindent definimos $p(x)$ como o grau de sociabilidade do nó $x$, $netos(x)$ 
como a operação que devolve os netos do nó $x$ 
na árvore, e $filhos(x)$ a operação que devolve os nós filhos de $x$ na árvore. 
Também definimos que $netos(x) = \emptyset$ e $filhos(x) = \emptyset$ quando 
$x$ não tem nós netos ou filhos respectivamente.
\\[6pt]
\noindent Agora definimos a recorrência da seguinte forma:
\begin{equation*}
    c[x] =
    \begin{cases}
      0 & \text{,se } x = \emptyset \\
      p(x) & \text{,se $x$ é uma folha}  \\      
      \max{(p(x) + \sum_{y \in netos(x)}c[y], \sum_{y \in filhos(x)}c[y])}
                                              & \text{,caso contrário}
    \end{cases}
  \end{equation*}
 
\noindent \textbf{Algoritmo de programação dinâmica: } Abaixo segue o algoritmo 
de programação dinâmica.