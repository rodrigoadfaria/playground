
\noindent \textbf{16. UVA 10066} (Torres gêmeas) Havia em um império antigo duas torres de formatos diferentes, situadas cada uma em uma cidade diferente. As torres eram construídas de lajotas circulares postas uma sobre a outra. Cada uma das lajotas era da mesma altura e tinha um raio inteiro. Não era de se espantar no entanto, que apesar de terem formatos diferentes, as duas torres tinham muitas lajotas em comum.

Mais de mil anos depois que elas foram construídas, o imperador ordenou que seus arquitetos removessem algumas das lajotas das duas torres de maneira que elas ficassem com a mesma forma e tamanho, e que ao mesmo tempo ficassem tão altas quanto possível. A ordem das lajotas nas novas
torres deveria ser a mesma que nas torres originais. O imperador achou que, desta maneira, as torres seriam capazes de permanecer como símbolo da harmonia e igualdade entre as duas cidades. Ele decidiu chamá-las então de torres gêmeas.

Agora, cerca de dois mil anos depois, desafiamos você a resolver um problema mais simples: dada a descrição das duas torres distintas, você deve encontrar o número de lajotas que haveria no mais alto par de torres gêmeas que pode ser construído delas.\\[6pt]