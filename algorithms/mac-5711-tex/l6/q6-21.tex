
\noindent 21. (Bandeiras) No dia da Bandeira na Rússia o proprietário de uma loja decidiu decorar a vitrine de sua loja com faixas de tecido das cores branca, azul e vermelha. Ele deseja satisfazer as seguintes condições: faixas da mesma cor não podem ser colocadas uma ao lado da outra. Uma faixa azul sempre está entre uma branca e uma vermelha, ou uma vermelha e uma branca. Escreva um programa que, dado o número n de faixas a serem colocadas na vitrine, calcule o número de maneiras de satisfazer as condições do proprietário.

\textbf{Exemplo:} Para $n = 3$, o resultado são as seguintes combinações: BVB, VBV, BAV, VAB.\\[6pt]
\textbf{Resposta:} Podemos pensar no processo de formação da combinação das cores da seguinte maneira: a primeira posição, obrigatoriamente, deve conter um B ou um V. Vamos fazer uma simulação iniciando com B que, ao final, basta multiplicarmos por 2 para termos o número total de combinações. No segundo nível teremos então duas possibilidades: V e A.

Percebemos então que, a cada vez que temos um A, temos apenas uma possibilidade no nível seguinte. Quando temos um B ou um V, temos duas novas possibilidades. Devemos ter o cuidado ao finalizar as combinações para que não haja nós adjacentes com o mesmo valor e, também, nenhuma terminação em A.

% Set the overall layout of the tree
\tikzstyle{level 1}=[level distance=1.5cm, sibling distance=5cm]
\tikzstyle{level 2}=[level distance=1.5cm, sibling distance=3cm]
\tikzstyle{level 3}=[level distance=1.5cm, sibling distance=2cm]
\tikzstyle{level 4}=[level distance=1.5cm, sibling distance=1cm]
\tikzstyle{level 5}=[level distance=1.5cm, sibling distance=0.5cm]

% Define styles for bags and leafs
\tikzstyle{bag} = [shape=ellipse, draw, align=center,
                   top color=white, bottom color=gray!20]
\tikzstyle{end} = [shape=rectangle, rounded corners, draw, align=center, fill=gray!40]

\begin{figure}[!h]
\centering
\begin{tikzpicture}[grow=right]
\node[bag] {$B$}
    child {
        node[bag] {$A$}
            child {
                node[bag] {$V$}
                    child {
                        node[bag] {$A$}
                        child {
                            node[bag] {$B$}
                                child {
                                    node[bag] {$V$}
                                    edge from parent 
                                    node[above] {}
                    			}
            			}
        			}
                    child {
                        node[bag] {$B$}
                            child {
                                node[bag] {$A$}
                                    child {
                                        node[bag] {$V$}
                                        edge from parent 
                                        node[above] {}
                        			}
                			}
                            child {
                                node[bag] {$V$}
                                    child {
                                        node[bag] {$B$}
                                        edge from parent 
                                        node[above] {}
                        			}
                			}
        			}
			}
    }
    child {
        node[bag] {$V$}
            child {
                node[bag] {$A$}
                    child {
                        node[bag] {$B$}
                            child {
                                node[bag] {$A$}
                                    child {
                                        node[bag] {$V$}
                                        edge from parent 
                                        node[above] {}
                        			}
                			}
                            child {
                                node[bag] {$V$}
                                    child {
                                        node[bag] {$B$}
                                        edge from parent 
                                        node[above] {}
                        			}
                			}
        			}
			}
            child {
                node[bag] {$B$}
                    child {
                        node[bag] {$A$}
                            child {
                                node[bag] {$V$}
                                    child {
                                        node[bag] {$B$}
                                        edge from parent 
                                        node[above] {}
                        			}
                			}
        			}
                    child {
                        node[bag] {$V$}
                            child {
                                node[bag] {$A$}
                                    child {
                                        node[bag] {$B$}
                                        edge from parent 
                                        node[above] {}
                        			}
                			}
                            child {
                                node[bag] {$B$}
                                    child {
                                        node[bag] {$V$}
                                        edge from parent 
                                        node[above] {}
                        			}
                			}
        			}
			}
    };
\end{tikzpicture}
\captionof{figure}{Árvore de combinações possíveis para $n = 6$}
\label{fig:6.21-1}
\end{figure}

Logo, percebemos que o número de combinações de $1..n-1$ cresce conforme a série de Fibonacci e, portanto, podemos calcular o número de maneiras de satisfazer as condições do proprietário com a própria recorrência que calcula a série de Fibonacci:
\begin{align*}
t(n) = \left\{\begin{array}{rl}
                    0, &\mbox{se $n = 0$} \\
                    1, &\mbox{se $n = 1$} \\
                    t(n - 1) + t(n - 2) + 1, &\mbox{se $n \geq 2$}
                \end{array} \right.
\end{align*}

Como temos duas possibilidades (iniciando a série com B ou V), basta multiplicarmos o resultado da série de Fibonacci(n) por 2 e teremos, então, o número de diferentes maneiras que podemos organizar as faixas de tecido na vitrine. O algoritmo \proc{Arrange-Flags} mostra a solução que, na verdade, é uma pequena alteração do algoritmo original de Fibonacci para programação dinâmica.

\begin{codebox}
\Procname{$\proc{Arrange-Flags}(n)$}
\li $f[0] \gets 0$
\li $f[1] \gets 1$
\li \For $i \gets 2$ \To $n$
\li \Do
        $f[i] \gets f[i-1] + f[i-2]$
    \End
\li \Return 2 * f[n]
\end{codebox}

Sendo assim, no exemplo supra citado, para $n = 6$, temos que o número de combinações possíveis é 16.

Obviamente o consumo de espaço e tempo do algoritmo é $\Theta(n)$, assim como é o original.\\[12pt]