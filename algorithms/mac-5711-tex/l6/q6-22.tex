
\noindent \textbf{3. mac338-2008, p2} Dado $n$ e uma cadeia de $n$ caracteres $s[1..n]$ que você acredita ser um texto corrompido, em que toda a pontuação foi removida (de modo que pareça com alguma coisa assim... "eraumavezumgatoxadrez..."). Você deseja reconstruir o documento, usando um dicionário, que está disponível na forma de uma função booleana $dict(.)$ para cada cadeia de caracteres w,

\begin{equation*}
    dict(.) =
    \begin{cases}
        true & \text{se $w$ é uma palavra válida} \\
        false & \text{caso contrário}
    \end{cases}
\end{equation*}

Escreva um algoritmo de programação dinâmica que, dado $n$ e uma cadeia de caracteres $s[1..n]$, determina se $s$ pode ser reconstituída como uma sequência de palavras válidas. O seu algoritmo deve consumir tempo $O(n^2)$. Justifique porque ele funciona (por exemplo explicando a validade da recorrência de onde ele foi derivado) e porque o seu consumo de tempo é $O(n^2)$.

Neste caso, como a tabela com as possíveis palavras válidas já foi dada, basta consultarmos a tabela com um algoritmo memoizado.

A subestrutura ótima é, se de $s[i..k]$ temos uma palavra válida e de $s[k+1..j]$ também, então, temos que $s[i..j]$ pode ser reconstituída como uma sequência de palavras válidas.

A recorrência fica, então, da seguinte maneira:
\begin{equation*}
    dp[i, j] =
    \begin{cases}
        0 & \text{, se } i = j - 1 \text{ e } dict(s[i, j]) = false \\
        1 & \text{, se } dict(s[i, j]) \\
        \smash{\displaystyle\max_{i \leq k \leq j}} \{dp[i, k], dp[k+1, j]\} & \text{, c.c.}
    \end{cases}
\end{equation*}

O algoritmo \proc{Reconstitui} mostra a implementação da recorrência, sendo que as subsequências são consultadas por $dict(.)$ de forma memoizada.

\begin{codebox}
\Procname{$\proc{Reconstitui}(s, i, j)$}
\li \If $\proc{dict}(s[i:j])$
\li \Then
        \Return true
\li \Else
        \For $k \gets i$ \To $j$
\li     \Do
            \If $\proc{Reconstitui}(s, i, k)$ and $\proc{Reconstitui}(s, k+1, j)$
\li         \Then
                \Return true
            \End
        \End
    \End
\li \Return false
\end{codebox}

Como nós temos $n$ subproblemas e $n$ possibilidades no máximo, o consumo de tempo será $O(n^2)$, já que a cada chamada recursiva nós gastamos tempo constante.\\[12pt]