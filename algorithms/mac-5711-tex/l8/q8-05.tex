
\noindent\textbf{5. (CRLS 23.1-3)} Prove ou desprove a seguinte afirmação: Se uma aresta está contida em alguma MST, então tem peso mínimo dentre todas as arestas de algum corte no grafo.\\[6pt]
\textbf{Resposta:} Seja $T$ uma MST que contém a aresta $(u, v)$. Se tirarmos $(u, v)$ da árvore, teremos um corte $(S, V - S)$ que particiona os vértices em dois conjuntos disjuntos $S$ e $V - S$. A aresta $(u, v)$ atravessa esse corte e ela é uma aresta de custo mínimo pois, do contrário, outra aresta de custo menor poderia ser adicionada em $T$ substituindo $(u, v)$, o que produziria uma outra MST $T'$, onde $w(T') < w(T)$, contradizendo o fato de que $T$ é uma MST.

Portanto, $(u, v)$ é uma aresta de custo mínimo para o corte $(S, V - S)$.\\[6pt]