
\indent\textbf{15.} O diâmetro de um grafo é o máximo das distâncias entre dois vértices. Escreva código que usa o algoritmo de Dijkstra para calcular o diâmetro de um grafo.\\[6pt]
\textbf{Resposta:} Para calcular o diâmetro de um grafo $G$ podemos aplicar o \proc{Dijkstra} usando cada vértice $v \in V[G]$ como fonte da busca, retornando o maior valor de $d$ para cada busca. A maior distância será, então, o diâmetro de $G$.

\begin{codebox}
\Procname{$\proc{Graph-Diameter}(G, w)$}
\li $diameter \gets 0$
\li \For each $u \in V[G]$
\li \Do
        $max\_d = \proc{Dijkstra-Diameter}(G, w, u)$
\li     \If $max\_d > diameter$
\li     \Then 
            $diameter \gets max\_d$
        \End
    \End
\li \Return $diameter$
\end{codebox}

\begin{codebox}
\Procname{$\proc{Dijkstra-Diameter}(G, w, s)$}
\li $\proc{Initialize-Single-Source}(G, s)$
\li $S \gets \emptyset$
\li $Q \gets V[G]$
\li \While $Q \neq \emptyset$
\li \Do
        $u \gets \proc{Extract-Min}(Q)$
\li     $S \gets S \cup \{u\}$
\li     \For each $v \in Adj[u]$
\li     \Do
            $\proc{Relax}(u, v, w)$
        \End
    \End
\li \Return $u.\id{d}$ \Comment the last $u$ removed from $Q$
\end{codebox}

\textbf{Consumo de Tempo:} A inclusão da linha 9 no \proc{Dijkstra-Diameter} não muda o consumo de tempo do \proc{Dijkstra}, que continua $O(E lg V)$. Como executamos uma vez para cada vértice no \textit{loop} da linha 2 da rotina $\proc{Graph-Diameter}$, temos que o tempo total será $O(V(E lg V))$.\\[6pt]