
\noindent\textbf{16.} Considere um dígrafo (grafo orientado) com custos positivos associados aos vértices. O custo de um caminho num tal dígrafo é a soma dos custos dos vértices do caminho. Queremos encontrar um caminho de custo mínimo dentre os que começam num vértice $s$ e terminam num vértice $t$. Adapte o algoritmo de Dijkstra para resolver esse problema.\\[6pt]
\textbf{Resposta:} Basta pararmos o algoritmo de Dijkstra quando o vértice $t$ for encontrado. Como a estratégia do algoritmo é gulosa, ao retirarmos o vértice $t$ da fila de prioridade, já teremos o caminho mínimo de $s$ a $t$.

\begin{codebox}
\Procname{$\proc{Dijkstra-Single-Pair}(G, w, s, t)$}
\li $\proc{Initialize-Single-Source}(G, s)$
\li $S \gets \emptyset$
\li $Q \gets V[G]$
\li \While $Q \neq \emptyset$
\li \Do
        $u \gets \proc{Extract-Min}(Q)$
\li     $S \gets S \cup \{u\}$
\li     \If $u == t$
\li     \Then
            \Return
        \End
\li     \For each $v \in Adj[u]$
\li     \Do
            $\proc{Relax}(u, v, w)$
        \End
    \End
\end{codebox}

\textbf{Consumo de Tempo:} A inclusão das linhas 7-8 não muda o comportamento assintótico original do \proc{Dijkstra}, que continua que continua $O(E lg V)$.\\[6pt]