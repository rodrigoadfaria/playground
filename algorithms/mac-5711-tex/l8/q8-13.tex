\\[6pt]
\noindent \textbf{13 (CLRS Ex. 23.2-4,5)} Suponha que todos os pesos num grafo com $n$ vértices são inteiros no intervalo de 1 até $n$. Descreva como otimizar os algoritmos de Kruskal e Prim nesta situação. O que acontence se pos pesos são intervalo de 1 até $W$? 
\\[6pt]
\noindent \textbf{Resposta: } Para o algoritmo de Kruskal utilizando a estrutura de conjuntos disjuntos utilizando $\proc{Union-By-Rank}$ e compressão de caminhos o tempo gasto assintoticamente é definido pela ordenação de arestas, no começo do algoritmo, se sabemos que o peso máximo das arestas é $n$, então podemos usar o $\proc{Counting-Sort}$ e diminuir o tempo de $O(m \lg n)$ para $O(m + n)$, já que o laço abaixo da ordenação gasta tempo linear, se o peso máximo é $W$ então utilizando o $\proc{Counting-Sort}$ o algoritmo de Kruskal gastará tempo $O(m + W)$. Como o algoritmo de Prim utiliza uma fila de prioridades que normalmente é implementada como uma min-heap não é possível melhorar o tempo assintotico do algoritmo.