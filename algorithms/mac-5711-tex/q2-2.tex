\noindent 2. Escreva um algoritmo que ordena uma lista de n itens dividindo-a em três sublistas de aproximadamente $n/3$ itens, ordenando cada sublista recursivamente e intercalando as três sublistas ordenadas. Analise seu algoritmo concluindo qual é o seu consumo de tempo.\\[6pt]
Para este exercício, devemos efetuar uma alteração no $\proc{Mergesort}$ para a divisão do vetor $A$ em três partições utilizando o $\proc{Merge}$ duas vezes ao final para intercalar as três partes ordenadas em um único vetor.

\begin{codebox}
\Procname{$\proc{Mergesort3}(A, p, r)$}
\li    \If $p < r$
\li         \Then
            $k = \floor{(p + r) / 3}$
\li         $m = k+1 + \floor{(p + r) / 3}$
\li         $\proc{Mergesort3}(A, p, k)$
\li         $\proc{Mergesort3}(A, k + 1, m)$
\li         $\proc{Mergesort3}(A, m + 1, r)$
\li         $\proc{Merge}(A, p, k, m)$
\li         $\proc{Merge}(A, p, m, r)$
            \End
\End
\end{codebox}

\textbf{Consumo de tempo}\\[6pt]
As linhas 1-3 consomem $\Theta(1)$. As linhas 4-5 têm consumo $T(\ceil{n / 3})$ e a linha 6 tem consumo $T(n - \ceil{2n / 3})$, já que a terceira partição não tem tamanho exatamente de $\ceil{n / 3}$. Sabemos que o consumo do $\proc{Merge}$ é $\Theta(n)$, logo:\\
\begin{align*}
T(n) & = T(\ceil{n / 3}) + T(\ceil{n / 3}) + T(n - 2\ceil{n / 3}) + \Theta(n) + \Theta(n) \\
& = 2T(\ceil{n / 3}) + T(\ceil{n / 3}) + \Theta(2n) \\
& = 3T(\ceil{n / 3}) + \Theta(2n)
\end{align*}

Como $\Theta(2n)$ é $\Theta(n)$:\\
\begin{align*}
T(n) & = 3T(\ceil{n / 3}) + \Theta(n)
\end{align*}

Simplificando a recorrência, temos:\\
\begin{eqnarray*}
T(n) = \left\{ \begin{array}{rl} 
 1, &\mbox{ $n = 1$} \\
 3T\bgfrac{n}{3} + n, &\mbox{ $n >= 2 $ potência de 2}
       \end{array} \right.
\end{eqnarray*}

Por expansão:\\
\begin{align*}
T(n) & = 3T\bgfrac{n}{3} + n\\
&= 3 \bigg( 3T\bgfrac{n}{3^2} + \bgfrac{n}{3} \bigg) + n &= 3^2T\bgfrac{n}{3^2} + n + n\\
&= 3^2 \bigg( 3T\bgfrac{n}{3^3} + \bgfrac{n}{3^2} \bigg) + n + n &= 3^3T\bgfrac{n}{3^3} + n + n + n\\
&= ... \\
&= 3^kT\bgfrac{n}{3^k} + kn
\end{align*}

Assumindo $k = \log_3{n}$ e $3^k = n$:\\
\begin{align*}
T(n) &= nT\bgfrac{n}{n} + \log_3{n}\\
&= T(1)n + \log_3{n}(n)\\
&= n + n(\log_3{n})\\
\end{align*}

Portanto, $T(n) = n + n(\log_3{n})$ é $\Theta(n\log{n})$.

\begin{proof}
Prova por indução em k.

\textbf{Base:} para $n = 1$
\begin{align*}
T(1) = 1 = 1 + 1(\log_3{1}) = 1+ 0 = 1
\end{align*}

\textbf{Hipótese de Indução:} Assuma que $T(x) = x + x(\log_3{x})$ vale para $1 >= x < n$\\

\textbf{Passo:} para $n >= 2$
\begin{align*}
T(n) = 3T\bgfrac{n}{3} + n &= 3\bigg( \bgfrac{n}{3} + \bgfrac{n(\log_3{\frac{n}{3}})}{3} \bigg) + n & (\text{por HI})\\
&= 3\bgfrac{n}{3} + 3\bigg( \bgfrac{n}{3}\log_3{\frac{n}{3}}\bigg) + n\\
&= n + n + n\log_3{\frac{n}{3}}\\
&= 2n + n\log_3{n} - n\\
&= n + n\log_3{n}
\end{align*}

Como queríamos demonstrar!

\end{proof}