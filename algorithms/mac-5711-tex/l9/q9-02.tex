
\noindent\textbf{2.} Mostre que SAT está em NP. (Essa é a parte fácil do teorema de Cook.)\\[6pt]
\textcolor{red}{\textbf{Resposta:}} Para mostrar que o SAT está em NP, temos que mostrar que existe um algoritmo $A(C,X,Y)$ que verifica o certificado $Y$ que devolve SIM se $Y$ pertence ao conjuto de soluções para a instância $C, X$.

Seja $C$ um conjunto de clasulas sobre um conjunto de variáveis $X$, então podemos considerar $C$ e $X$  nossa instância do problema SAT. Seja $Y$ uma atribuição das variáveis de $X$, podemos considerar $Y$ como nosso certificado, já que se as clasulas $C$ forem satisfazivel por $Y$ então a formula é satisfazivel. É fácil notar que podemos escrever um algoritmo que leia $X,C$ e $Y$ e imprime SIM se e somente se $Y$ satisfaz $C$, álem disso esse algoritmo pode ser implementado de forma a consumir tempo polinomial em $|X|$ e $|C|$, (note que $|Y| = |X|$). Portanto SAT pode ser verificado e certificado em tempo polinomial, logo SAT está em NP.