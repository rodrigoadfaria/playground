% --------------------------------------------------------------
% This is all preamble stuff that you don't have to worry about.
% Head down to where it says "Start here"
% --------------------------------------------------------------
 
\documentclass[12pt]{article}

\usepackage[latin1,utf8]{inputenc}
\usepackage[brazil]{babel}

\usepackage[margin=1in]{geometry} 
\usepackage{amsmath,amsthm,amssymb}
\usepackage{clrscode3e} % for the pseudocode

\newcommand{\N}{\mathbb{N}}
\newcommand{\Z}{\mathbb{Z}}
 
\newenvironment{theorem}[2][Theorem]{\begin{trivlist}
\item[\hskip \labelsep {\bfseries #1}\hskip \labelsep {\bfseries #2.}]}{\end{trivlist}}
\newenvironment{lemma}[2][Lemma]{\begin{trivlist}
\item[\hskip \labelsep {\bfseries #1}\hskip \labelsep {\bfseries #2.}]}{\end{trivlist}}
\newenvironment{exercise}[2][Exercise]{\begin{trivlist}
\item[\hskip \labelsep {\bfseries #1}\hskip \labelsep {\bfseries #2.}]}{\end{trivlist}}
\newenvironment{reflection}[2][Reflection]{\begin{trivlist}
\item[\hskip \labelsep {\bfseries #1}\hskip \labelsep {\bfseries #2.}]}{\end{trivlist}}
\newenvironment{proposition}[2][Proposition]{\begin{trivlist}
\item[\hskip \labelsep {\bfseries #1}\hskip \labelsep {\bfseries #2.}]}{\end{trivlist}}
\newenvironment{corollary}[2][Corollary]{\begin{trivlist}
\item[\hskip \labelsep {\bfseries #1}\hskip \labelsep {\bfseries #2.}]}{\end{trivlist}}

\newcommand\floor[1]{\big\lfloor#1\big\rfloor}
\newcommand\ceil[1]{\big\lceil#1\big\rceil}


\begin{document}
 
% --------------------------------------------------------------
%                         Start here
% --------------------------------------------------------------
 
%\renewcommand{\qedsymbol}{\filledbox}
 
\title{MAC 5711 - Análise de Algoritmos}
\author{Rodrigo Augusto Dias Faria\\
Departamento de Ciência da Computação - IME/USP}
 
\maketitle
 
\begin{theorem}{x.yz} %You can use theorem, proposition, exercise, or reflection here.  Modify x.yz to be whatever number you are proving
Delete this text and write theorem statement here.
\end{theorem}
 
\begin{proof}
Blah, blah, blah.  Here is an example of the \texttt{align} environment:
%Note 1: The * tells LaTeX not to number the lines.  If you remove the *, be sure to remove it below, too.
%Note 2: Inside the align environment, you do not want to use $-signs.  The reason for this is that this is already a math environment. This is why we have to include \text{} around any text inside the align environment.
\begin{align*}
\sum_{i=1}^{k+1}i & = \left(\sum_{i=1}^{k}i\right) +(k+1)\\ 
& = \frac{k(k+1)}{2}+k+1 & (\text{by inductive hypothesis})\\
& = \frac{k(k+1)+2(k+1)}{2}\\
& = \frac{(k+1)(k+2)}{2}\\
& = \frac{(k+1)((k+1)+1)}{2}.
\end{align*}
\end{proof}
 
\begin{proposition}{x.yz}
Let $n\in \Z$.  
\end{proposition}
 
\begin{proof}[Disproof]%Whatever you put in the square brackets will be the label for the block of text to follow in the proof environment.
Blah, blah, blah.  I'm so smart.
\end{proof}

\clearpage
\begin{center} 
\textbf{\large{Lista 3}}
\end{center}

\noindent 2. Qual é o consumo de espaço do QUICKSORT no pior caso?\\[6pt]
A avaliação de um algoritmo quanto ao consumo de espaço está relacionada com a necessidade de alocação de espaço adicional na pilha de recursão.

No pior caso, o QUICKSORT será executado uma vez para cada elemento da lista dada de tamanho $n$, ou seja, teremos $n$ chamadas recursivas.

Isso significa que, com uma lista de $n$ elementos, $n$ novas chamadas serão adicionadas à pilha no pior caso, o que nos leva a uma complexidade de espaço O($n$).
\clearpage

\begin{center} 
\textbf{\large{Lista 4}}
\end{center}

\noindent 1. Escreva uma função que recebe um vetor com n letras A’s e B’s e, por meio de trocas, move todos os A’s para o início do vetor. Sua função deve consumir tempo O($n$).\\[6pt]
Resposta\\
\input{q4-2}

\noindent 3. Sejam $X[1..n]$ e $Y[1..n]$ dois vetores, cada um contendo $n$ números ordenados. Escreva um algoritmo O(lg $n$) para encontrar uma das medianas de todos os $2n$ elementos nos vetores $X$ e $Y$.\\[6pt]
Sabemos que a mediana de X e Y está em $i = \floor{q / 2}$ e $j = \floor{s / 2}$, respectivamente. Note que $n = q + s$ é par, e é por isso que nós estamos usando a função \textbf{piso}.

Se $X[i]$ é maior do que $Y[j]$, significa que a mediana global está à esquerda de $X[i]$ e à direita de $Y[j]$. Se $X[i]$ é menor ou igual a $Y[j]$, nós procuramos a mediana à esquerda de $Y[j]$ e à direita de $X[i]$.

A condição de parada dá-se quando $p == q$, o que significa que a mediana global está dentro do vetor $X$. Caso contrário, se $r == s$, a mediana está em $Y$.

O pseudocódigo $\proc{Find-Median}$ mostra a operação descrita acima que, também, é o resultado do exercício 9.3-8 CLRS 3ed.\\

\begin{codebox}
\Procname{$\proc{Find-Median}(X, Y, p, q, r, s)$}
\li \If $p \isequal q$
\li \Comment We have found the median between p, q and r
\li     \Then
            \Return $X[p]$
\li     \ElseIf $r \isequal s$
\li \Comment We have found the median between q, r and s
\li     \Then
            \Return $Y[r]$
        \End
\li $i \gets p + (q - p) / 2$
\li $j = r + (s - r) / 2$
\li \If $X[i] > Y[j]$
\li     \Then
            $q \gets i$
\li         $r \gets j$
\li     \Else
\li         $p \gets i$
\li         $s \gets j$
        \End
\li \Return $\proc{Find-Median}(X, Y, p, q, r, s)$
\End
\end{codebox}

\end{document}