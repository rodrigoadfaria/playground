% --------------------------------------------------------------
% This is all preamble stuff that you don't have to worry about.
% Head down to where it says "Start here"
% --------------------------------------------------------------
 
\documentclass[12pt]{article}

\usepackage[latin1,utf8]{inputenc}
\usepackage[brazil]{babel}

\usepackage[margin=1in]{geometry} 
\usepackage{amsmath,amsthm,amssymb}
\usepackage{clrscode3e} % for the pseudocode

\newcommand{\N}{\mathbb{N}}
\newcommand{\Z}{\mathbb{Z}}
 
\newenvironment{theorem}[2][Theorem]{\begin{trivlist}
\item[\hskip \labelsep {\bfseries #1}\hskip \labelsep {\bfseries #2.}]}{\end{trivlist}}
\newenvironment{lemma}[2][Lemma]{\begin{trivlist}
\item[\hskip \labelsep {\bfseries #1}\hskip \labelsep {\bfseries #2.}]}{\end{trivlist}}
\newenvironment{exercise}[2][Exercise]{\begin{trivlist}
\item[\hskip \labelsep {\bfseries #1}\hskip \labelsep {\bfseries #2.}]}{\end{trivlist}}
\newenvironment{reflection}[2][Reflection]{\begin{trivlist}
\item[\hskip \labelsep {\bfseries #1}\hskip \labelsep {\bfseries #2.}]}{\end{trivlist}}
\newenvironment{proposition}[2][Proposition]{\begin{trivlist}
\item[\hskip \labelsep {\bfseries #1}\hskip \labelsep {\bfseries #2.}]}{\end{trivlist}}
\newenvironment{corollary}[2][Corollary]{\begin{trivlist}
\item[\hskip \labelsep {\bfseries #1}\hskip \labelsep {\bfseries #2.}]}{\end{trivlist}}

\newcommand\floor[1]{\big\lfloor#1\big\rfloor}
\newcommand\ceil[1]{\big\lceil#1\big\rceil}
\newcommand\bgfrac[2]{\bigg(\dfrac{#1}{#2}\bigg)}


\begin{document}
 
% --------------------------------------------------------------
%                         Start here
% --------------------------------------------------------------
 
%\renewcommand{\qedsymbol}{\filledbox}
 
\title{MAC 5711 - Análise de Algoritmos}
\author{Rodrigo Augusto Dias Faria\\
Departamento de Ciência da Computação - IME/USP}
 
\maketitle
 
\begin{proof}
Blah, blah, blah.  Here is an example of the \texttt{align} environment:

\end{proof}
 
\begin{proof}[Disproof]%Whatever you put in the square brackets will be the label for the block of text to follow in the proof environment.
Blah, blah, blah.  I'm so smart.
\end{proof}

\clearpage
\begin{center} 
\textbf{\large{Lista 1}}
\end{center}
\noindent 1. Lembre-se que lg $n$ denota o logaritmo na base 2 de $n$. Usando a definição de notação $O$, prove que\\[6pt]
(a) $3n$ não é $O(2^n)$\\
(b) $\log_{10}n$ é $O$(lg $n$)\\
(c) lg $n$ é $O(\log_{10}n)$\\

\clearpage
\begin{center} 
\textbf{\large{Lista 2}}
\end{center}
\noindent 2. Escreva um algoritmo que ordena uma lista de n itens dividindo-a em três sublistas de aproximadamente $n/3$ itens, ordenando cada sublista recursivamente e intercalando as três sublistas ordenadas. Analise seu algoritmo concluindo qual é o seu consumo de tempo.\\[6pt]
Para este exercício, devemos efetuar uma alteração no $\proc{Mergesort}$ para a divisão do vetor $A$ em três partições utilizando o $\proc{Merge}$ duas vezes ao final para intercalar as três partes ordenadas em um único vetor.

\begin{codebox}
\Procname{$\proc{Mergesort3}(A, p, r)$}
\li    \If $p < r$
\li         \Then
            $k = \floor{(p + r) / 3}$
\li         $m = k+1 + \floor{(p + r) / 3}$
\li         $\proc{Mergesort3}(A, p, k)$
\li         $\proc{Mergesort3}(A, k + 1, m)$
\li         $\proc{Mergesort3}(A, m + 1, r)$
\li         $\proc{Merge}(A, p, k, m)$
\li         $\proc{Merge}(A, p, m, r)$
            \End
\End
\end{codebox}

\textbf{Consumo de tempo}\\[6pt]
As linhas 1-3 consomem $\Theta(1)$. As linhas 4-5 têm consumo $T(\ceil{n / 3})$ e a linha 6 tem consumo $T(n - \ceil{2n / 3})$, já que a terceira partição não tem tamanho exatamente de $\ceil{n / 3}$. Sabemos que o consumo do $\proc{Merge}$ é $\Theta(n)$, logo:\\
\begin{align*}
T(n) & = T(\ceil{n / 3}) + T(\ceil{n / 3}) + T(n - 2\ceil{n / 3}) + \Theta(n) + \Theta(n) \\
& = 2T(\ceil{n / 3}) + T(\ceil{n / 3}) + \Theta(2n) \\
& = 3T(\ceil{n / 3}) + \Theta(2n)
\end{align*}

Como $\Theta(2n)$ é $\Theta(n)$:\\
\begin{align*}
T(n) & = 3T(\ceil{n / 3}) + \Theta(n)
\end{align*}

Simplificando a recorrência, temos:\\
\begin{eqnarray*}
T(n) = \left\{ \begin{array}{rl} 
 1, &\mbox{ $n = 1$} \\
 3T\bgfrac{n}{3} + n, &\mbox{ $n >= 2 $ potência de 2}
       \end{array} \right.
\end{eqnarray*}

Por expansão:\\
\begin{align*}
T(n) & = 3T\bgfrac{n}{3} + n\\
&= 3 \bigg( 3T\bgfrac{n}{3^2} + \bgfrac{n}{3} \bigg) + n &= 3^2T\bgfrac{n}{3^2} + n + n\\
&= 3^2 \bigg( 3T\bgfrac{n}{3^3} + \bgfrac{n}{3^2} \bigg) + n + n &= 3^3T\bgfrac{n}{3^3} + n + n + n\\
&= ... \\
&= 3^kT\bgfrac{n}{3^k} + kn
\end{align*}

Assumindo $k = \log_3{n}$ e $3^k = n$:\\
\begin{align*}
T(n) &= nT\bgfrac{n}{n} + \log_3{n}\\
&= T(1)n + \log_3{n}(n)\\
&= n + n(\log_3{n})\\
\end{align*}

Portanto, $T(n) = n + n(\log_3{n})$ é $\Theta(n\log{n})$.

\begin{proof}
Prova por indução em k.

\textbf{Base:} para $n = 1$
\begin{align*}
T(1) = 1 = 1 + 1(\log_3{1}) = 1+ 0 = 1
\end{align*}

\textbf{Hipótese de Indução:} Assuma que $T(x) = x + x(\log_3{x})$ vale para $1 >= x < n$\\

\textbf{Passo:} para $n >= 2$
\begin{align*}
T(n) = 3T\bgfrac{n}{3} + n &= 3\bigg( \bgfrac{n}{3} + \bgfrac{n(\log_3{\frac{n}{3}})}{3} \bigg) + n & (\text{por HI})\\
&= 3\bgfrac{n}{3} + 3\bigg( \bgfrac{n}{3}\log_3{\frac{n}{3}}\bigg) + n\\
&= n + n + n\log_3{\frac{n}{3}}\\
&= 2n + n\log_3{n} - n\\
&= n + n\log_3{n}
\end{align*}

Como queríamos demonstrar!

\end{proof}

\clearpage
\begin{center} 
\textbf{\large{Lista 3}}
\end{center}

\noindent 2. Qual é o consumo de espaço do QUICKSORT no pior caso?\\[6pt]
A avaliação de um algoritmo quanto ao consumo de espaço está relacionada com a necessidade de alocação de espaço adicional na pilha de recursão.

No pior caso, o QUICKSORT será executado uma vez para cada elemento da lista dada de tamanho $n$, ou seja, teremos $n$ chamadas recursivas.

Isso significa que, com uma lista de $n$ elementos, $n$ novas chamadas serão adicionadas à pilha no pior caso, o que nos leva a uma complexidade de espaço O($n$).
\clearpage

\begin{center} 
\textbf{\large{Lista 4}}
\end{center}

\noindent 1. Escreva uma função que recebe um vetor com n letras A’s e B’s e, por meio de trocas, move todos os A’s para o início do vetor. Sua função deve consumir tempo O($n$).\\[6pt]
Resposta\\
\input{q4-2}

\noindent 3. Sejam $X[1..n]$ e $Y[1..n]$ dois vetores, cada um contendo $n$ números ordenados. Escreva um algoritmo O(lg $n$) para encontrar uma das medianas de todos os $2n$ elementos nos vetores $X$ e $Y$.\\[6pt]
Sabemos que a mediana de X e Y está em $i = \floor{q / 2}$ e $j = \floor{s / 2}$, respectivamente. Note que $n = q + s$ é par, e é por isso que nós estamos usando a função \textbf{piso}.

Se $X[i]$ é maior do que $Y[j]$, significa que a mediana global está à esquerda de $X[i]$ e à direita de $Y[j]$. Se $X[i]$ é menor ou igual a $Y[j]$, nós procuramos a mediana à esquerda de $Y[j]$ e à direita de $X[i]$.

A condição de parada dá-se quando $p == q$, o que significa que a mediana global está dentro do vetor $X$. Caso contrário, se $r == s$, a mediana está em $Y$.

O pseudocódigo $\proc{Find-Median}$ mostra a operação descrita acima que, também, é o resultado do exercício 9.3-8 CLRS 3ed.\\

\begin{codebox}
\Procname{$\proc{Find-Median}(X, Y, p, q, r, s)$}
\li \If $p \isequal q$
\li \Comment We have found the median between p, q and r
\li     \Then
            \Return $X[p]$
\li     \ElseIf $r \isequal s$
\li \Comment We have found the median between q, r and s
\li     \Then
            \Return $Y[r]$
        \End
\li $i \gets p + (q - p) / 2$
\li $j = r + (s - r) / 2$
\li \If $X[i] > Y[j]$
\li     \Then
            $q \gets i$
\li         $r \gets j$
\li     \Else
\li         $p \gets i$
\li         $s \gets j$
        \End
\li \Return $\proc{Find-Median}(X, Y, p, q, r, s)$
\End
\end{codebox}
\noindent 4. (\textbf{CLRS 9.3-5}) Para esta questão, vamos dizer que a mediana de um vetor $A[p..r]$ com números inteiros é o valor que ficaria na posição $A[\floor{(p + r)/2}]$ depois que o vetor $A[p..r]$ fosse ordenado.\\
Dado um algoritmo linear “caixa-preta” que devolve a mediana de um vetor, descreva um algoritmo simples, linear, que, dado um vetor $A[p..r]$ de inteiros distintos e um inteiro $k$, devolve o \textit{k-ésimo} mínimo do vetor. (O \textit{k-ésimo} mínimo de um vetor de inteiros distintos é o elemento que estaria na \textit{k-ésima} posição do vetor se ele fosse ordenado.)\\[6pt]
Resposta\\

\noindent 8. (\textbf{CLRS 8.3-2}) Quais dos seguintes algoritmos de ordenação são estáveis: insertionsort, mergesort, heapsort, e quicksort. Descreva uma maneira simples de deixar qualquer algoritmo de ordenação estável. Quanto tempo e/ou espaço adicional a sua estratégia usa?\\[6pt]
Os algoritmos estáveis são o insertionsort e o mergesort (versão do Cormen). Os demais não são estáveis.

Uma forma simples de deixar qualquer algoritmo de ordenação estável é criar um mecanismo de indexação que mantenha a ordem em que os elementos aparecem originalmente, ou seja, basta termos um índice para cada elemento de um vetor de $n$ elementos.

Esse mecanismo necessita de $\Theta(n)$ espaço extra para armazenar os $n$ índices do vetor de $n$ elementos.\\


\end{document}