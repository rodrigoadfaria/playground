
\noindent\textbf{5.} Escreva uma versão não recursiva da busca em profundidade.

\textbf{Resposta:} Podemos utilizar uma pilha como apoio para aprofundar na lista de adjacências de cada nó da árvore, substituindo a recursão.

Além disso, nós usamos uma fila FIFO para a lista de adjacências de cada vértice $u$ e, assim que um novo vértice $v$ adjacente à $u$ é encontrado e ele ainda não foi visitado, nós o visitamos e empilhamos $v$ em $S$.

Note que nós somente retiramos $u$ da pilha (linha 11 da \proc{DFS-Visit}) quando todos os vértices da lista de adjacências dele foram visitados, ou seja, o momento em que devemos marcar $u$ como finalizado (sub-rotina \proc{Blacken}).

Outro ponto importante é que, como visto na linha 12 da \proc{Iterative-DFS}, o vértice $s$ que dá origem à busca tem seu ancestral como $nil$, o que garante o funcionamento da mesma forma que a \proc{Dfs} original. Isso também garante que a busca funciona nos casos em que tivermos mais de uma componente conexa no grafo.

\textbf{Consumo de tempo:} O \textit{loop} das linhas 2-8 da \proc{Iterative-DFS} tomam $\Theta(V)$ para inicializar cada vértice $u \in V[G]$ + $\Theta(E)$ para montar a fila de adjacências de cada vértice $u$. As sub-rotinas \proc{Blacken}, \proc{Grayen}, bem como as operações na fila/pilha tomamm $\Theta(1)$. O \textit{loop} das linhas 10-13 consome tempo $\Theta(V)$, ou seja, executa no máximo uma vez para cada vértice $v \in G$, já que \proc{DFS-Visit} é executado apenas em vértices que ainda não foram descobertos - àqueles que ainda são brancos. Como a fila de adjacências é visitada no máximo uma vez na sub-rotina \proc{DFS-Visit} e a soma do comprimento da fila de adjacências de cada vértice $v$ é $\Theta(E)$, o tempo gasto na \proc{DFS-Visit} é $O(E)$.

Portanto, o consumo de tempo total será $O(V + E)$, o que mantém o comportamento assintótico original da \proc{DFS}.

\begin{codebox}
\Procname{$\proc{Iterative-DFS}(G)$}
\li \Comment let $S$ be an empty stack
\li \For each vertex $u \in V[G]$
\li \Do
        $\id{color}[u] \gets \const{white}$
\li     $\id{\pi}[u] \gets \const{nil}$
\li     $n \gets \id{Adj}[u].length$
\li     \For $i \gets n$ to $1$
\li     \Do
            $v \gets \id{Adj}[u][i]$
\li         $\proc{Enqueue}(\id{Qadj}[u], v)$
        \End
    \End
\li $time \gets 0$
\li \For each vertex $u \in V[G]$
\li \Do
        \If $\id{color}[u] == \const{white}$
\li     \Then 
            $\proc{Grayen}(u, \const{nil})$
\li         $\proc{DFS-Visit}(u)$
    \End
\end{codebox}

\begin{codebox}
\Procname{$\proc{DFS-Visit}(s)$}
\li $\proc{Push}(S, s)$
\li \While $S \neq \emptyset$
\li \Do
        $u \gets \proc{Top}(S)$
\li     \If $\id{Qadj}[u] \neq \emptyset$
\li     \Then
            $v \gets \proc{Dequeue}(\id{Qadj}[u])$
\li         \If $\id{color}[v] == \const{white}$
\li         \Then
                $\proc{Grayen}(v, u)$
\li             $\proc{Push}(S, v)$
\li         \End
        \Else
\li         $\proc{Blacken}(u)$
\li         $\proc{Pop}(S)$
        \End
    \End
\end{codebox}

\begin{codebox}
\Procname{$\proc{Grayen}(v, u)$}
\li $\id{color}[v] \gets \const{gray}$
\li $time \gets time + 1$
\li $\id{d}[v] \gets time$
\li $\id{\pi}[v] \gets u$
\end{codebox}

\begin{codebox}
\Procname{$\proc{Blacken}(u)$}
\li $\id{color}[u] \gets \const{black}$
\li $time \gets time + 1$
\li $\id{f}[u] \gets time$
\end{codebox}