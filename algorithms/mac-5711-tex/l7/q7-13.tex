
\textbf{13.} Escreva uma generalização comum das buscas em largura e em profundidade. Sua função deve usar uma estrutura de dados auxiliar que pode operar como fila ou como pilha. Se a estrutura operar como fila, a função executa busca em largura, e se operar como pilha, a função executa busca em profundidade.\\[6pt]
\textbf{Resposta:} Basta efetuarmos algumas alterações na versão iterativa da \proc{Iteratve-DFS} para que a estrutura de dados opere de forma tal que ambas as buscas são atendidas. Basicamente, a ordem em que um vértice descoberto é incluído na pilha: se for busca em profundidade, incluímos no topo da pilha, se for em largura, incluímos na base (\proc{Push-Left}).

No caso de busca em largura, a sub-rotina \proc{Grayen} atualiza $d[v] = d[u] + 1$, considerando o caso onde o vértice é a origem da busca ($d[v] = 0$). Além disso, \proc{Blacken} só atualiza $f[u]$ se for uma busca em profundidade.

\textbf{Consumo de tempo:} As alterações não afetam o consumo de tempo obtido na \proc{Iterative-DFS}, permanecendo $O(V + E)$, o que mantém o comportamento original tanto da \proc{DFS}, quanto da \proc{BFS}.

\begin{codebox}
\Procname{$\proc{GFS}(G, type)$}
\li \Comment let $S$ be an empty stack
\li \For each vertex $u \in V[G]$
\li \Do
        $\id{color}[u] \gets \const{white}$
\li     $\id{\pi}[u] \gets \const{nil}$
\li     $n \gets \id{Adj}[u].length$
\li     \For $i \gets n$ to $1$
\li     \Do
            $v \gets \id{Adj}[u][i]$
\li         $\proc{Enqueue}(\id{Qadj}[u], v)$
        \End
    \End
\li $time \gets 0$
\li \For each vertex $u \in V[G]$
\li \Do
        \If $\id{color}[u] == \const{white}$
\li     \Then 
            $\proc{Grayen}(u, \const{nil})$
\li         $\proc{GFS-Visit}(u)$
    \End
\end{codebox}

\begin{codebox}
\Procname{$\proc{GFS-Visit}(s)$}
\li $\proc{Push}(S, s)$
\li \While $S \neq \emptyset$
\li \Do
        $u \gets \proc{Top}(S)$
\li     \If $\id{Qadj}[u] \neq \emptyset$
\li     \Then
            $v \gets \proc{Dequeue}(\id{Qadj}[u])$
\li         \If $\id{color}[v] == \const{white}$
\li         \Then
                $\proc{Grayen}(v, u)$
\li             \If $type == \const{bfs}$
\li             \Then
                    $\proc{Push-Left}(S, v)$
\li             \Else
\li                 $\proc{Push}(S, v)$
                \End
            \End
\li     \Else
\li         $\proc{Blacken}(u)$
\li         $\proc{Pop}(S)$
        \End
    \End
\end{codebox}

\begin{codebox}
\Procname{$\proc{Grayen}(v, u)$}
\li $\id{color}[v] \gets \const{gray}$
\li $\id{\pi}[v] \gets u$
\li \If $type == \const{bfs}$
\li \Then
        \If $u == \const{nil}$ \Comment it is a source vertex of a component
\li     \Then
            $\id{d}[v] \gets 0$
\li     \Else
\li         $\id{d}[v] \gets \id{d}[u] + 1$
        \End
\li \Else
\li     $time \gets time + 1$
\li     $\id{d}[v] \gets time$
    \End
\end{codebox}

\begin{codebox}
\Procname{$\proc{Blacken}(u)$}
\li $\id{color}[u] \gets \const{black}$
\li \If $type == \const{dfs}$
\li \Then
        $time \gets time + 1$
\li     $\id{f}[u] \gets time$
    \End
\end{codebox}