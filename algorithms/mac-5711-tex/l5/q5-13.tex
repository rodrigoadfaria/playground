
\noindent \textbf{2. mac338-2011, p2} (Esta questão é uma modificação de um dos exercícios de uma das listas.) Seja $x_1, x_2, \ldots , x_n$ uma sequência de números, onde $n$ é par. Um pareamento de $x_1, x_2, \ldots , x_n$ é uma partição do (multi)conjunto $\{x_1, x_2, \ldots , x_n\}$ em pares. Se P é um pareamento de $x_1, x_2, \ldots , x_n$, então a altura de P é o valor $max\{x_i+x_j : \{x_i, x_j\}$ é um par de $P\}$. Um pareamento de $x_1, x_2, \ldots , x_n$ é ótimo se tem altura mínima.\\[6pt]

\noindent (a) Escreva (em pseudo-código, como nas aulas) um algoritmo que, dado $n$ e um vetor $x[1..n]$ com $n$ números, encontre e devolva um pareamento ótimo de $x[1], x[2], \dots , x[n]$. Seu algoritmo deve
consumir tempo $O(n lg n)$.\\[2pt]
(b) Analise o consumo de tempo do seu algoritmo, concluindo que de fato é $O(n lg n)$.\\[2pt]
(c) Prove que ele de fato produz um pareamento ótimo, ou seja, que nenhum emparelhamento pode ter uma altura menor que a do emparelhamento produzido pelo seu algoritmo.

\textbf{Resposta:} Abaixo o algoritmo que retorna o pareamento ótimo. (resposta do gabarito):
\begin{codebox}
\Procname{$\proc{Pareamento}(x, n)$}
\li $\proc{Ordena}(x, n)$ \Comment em ordem crescente
\li \For $i \gets 1$ \To $n/2$
\li \Do
        $P[i] \gets \{x[i], x[n- i + 1]\}$
    \End
\li \Return P
\end{codebox}

É fácil ver que o algoritmo consome tempo $O(n lg n)$. Resta justificar porque ele produz um pareamento ótimo.

Tome um pareamento ótimo P* que inclua pares $P[1], \ldots ,P[i]$ para $i$ o maior possível. Se $i = n/2$, não há nada a provar: $P = P*$ e, portanto, $P$ é um pareamento ótimo.

Se $i < n/2$, então considere o pareamento $P'$, derivado de $P*$ da seguinte maneira. $Em P*$ os elementos $x[i + 1]$ e $x[n - (i + 1) + 1]$ não formam um par. Então sejam $j$ e $k$ tais que, em $P*$, temos os pares $\{x[i+1], x[j]\}$ e $\{x[n-(i+1)+1], x[k]\}$. Seja $P'$ o pareamento que coincide com $P*$ exceto pela troca entre $x[j]$ e $x[n - (i + 1) + 1]$ nos pares acima.

Qual é a altura do pareamento $P'$?
Observe que $i+1 < j < n-(i+1)+1$, logo $x[j] \leq x[n-(i+1)+1]$. Ou seja, a soma do segundo par (para onde foi $x[j]$) ou ficou igual ou diminuiu. Em fórmulas, $x[j]+x[k] \leq x[n-(i+1)+1]+x[k]$.

Por outro lado, como $x[i+1] \leq x[k]$, temos também que $x[i+1]+x[n-(i+1)+1] \leq x[k]+x[n-(i+1)+1]$. Ou seja, com certeza a altura de $P'$ é menor ou igual à altura de $P*$. Mas como $P*$ é um pareamento ótimo, essas alturas são iguais e $P'$ é, também, um pareamento ótimo.

Porém $P'$ tem um par a mais em comum com $P$ (coincidiria com $P$ até $i+1$ pelo menos), uma contradição à escolha de $P*$. Ou seja, esse caso não ocorre. Ocorre apenas o caso em que $i = n/2$, e portanto, $P$ é um pareamento ótimo.\\[12pt]