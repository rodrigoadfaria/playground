
\noindent \textbf{2. mac338-2011, p2} (Esta questão é uma modificação de um dos exercícios de uma das listas.) Seja $x_1, x_2, \ldots , x_n$ uma sequência de números, onde $n$ é par. Um pareamento de $x_1, x_2, \ldots , x_n$ é uma partição do (multi)conjunto $\{x_1, x_2, \ldots , x_n\}$ em pares. Se P é um pareamento de $x_1, x_2, \ldots , x_n$, então a altura de P é o valor $max\{x_i+x_j : \{x_i, x_j\}$ é um par de $P\}$. Um pareamento de $x_1, x_2, \ldots , x_n$ é ótimo se tem altura mínima.\\[6pt]

\noindent (a) Escreva (em pseudo-código, como nas aulas) um algoritmo que, dado $n$ e um vetor $x[1..n]$ com $n$ números, encontre e devolva um pareamento ótimo de $x[1], x[2], \dots , x[n]$. Seu algoritmo deve
consumir tempo $O(n lg n)$.\\[2pt]
(b) Analise o consumo de tempo do seu algoritmo, concluindo que de fato é $O(n lg n)$.\\[2pt]
(c) Prove que ele de fato produz um pareamento ótimo, ou seja, que nenhum emparelhamento pode ter uma altura menor que a do emparelhamento produzido pelo seu algoritmo.

\textcolor{red}{Tem no gabarito da prova}