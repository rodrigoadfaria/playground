\noindent 4. (\textbf{CLRS 8.1-3}) Mostre que não há algoritmo de ordenação 
baseado em comparações cujo consumo de tempo é linear para pelo menos metade de 
$n!$ permutações de 1 a $n$. O que acontece se trocarmos "metade" por uma 
fração 
de $1/n$? O que acontece se trocarmos "metade" por uma fração de $1/2^n$?
\\ [6pt]
\textbf{Resposta:} Assim como visto em aula uma árvore de decisão pode ser 
utilizada para representar o número de comparações executadas por um algoritmo. 
A árvore de decisão é uma árvore binária onde cada nó não folha é uma 
comparação 
e cada folha é uma permutação de entrada, o caminho da raiz da árvore até uma 
de 
suas folhas mostra a quantidade de comparações executadas para aquela 
permutação, portanto a distância da raiz para a folha mais distânte (altura da 
árvore) reflete o tempo gasto pelo pior caso do algoritmo.
\\[6pt]
Vamos verificar qual a altura da árvore de decisão para saber qual o tempo 
gasto 
no pior caso para pelo menos metade das possíveis permutações $n!/2$. Seja $l$ 
o 
número de folhas do algoritmo e $h$ a altura da árvore, como a árvore de 
decisão 
é uma árvore binária sabemos que ela terá no máximo $2^h$ folhas. Portanto 
temos 
a seguinte relação:
\[ \frac{n!}{2} \leq l \leq 2^h \]
Calculando o logaritmo dos dois lados e sabendo que logaritmo é uma função 
monotonicamente crescente temos que:
\begin{align*}
     \lg 2^h  &\geq  \lg \frac{n!}{2} \\
     h & \geq \lg n! - \lg 2 \\
     h & \geq \lg n! - 1 \\
\end{align*}
Como visto em aula $\lg n! \geq \frac{1}{2} n \lg n$, dessa inequação chegamos 
que $h \geq \lg n! \geq \frac{1}{2} n \lg n$. Disso podemos concluir que $h 
\geq 
\frac{1}{2} n \lg n - 1$, ou seja, mesmo utilizando apenas a metade das 
permutações assintoticamente o algoritmo ótimo gasta $\Omega(n \lg n)$ no pior 
caso.
\\[6pt]
Trocando a "metade" por uma fração $1/n$, podemos utilizar o mesmo raciocínio, 
definindo $l$ como a quantidade de folhas da árvore de decisão e $h$ a altura 
da 
árvore, teremos a seguinte inequação:
\[ \frac{n!}{n} \leq l \leq 2^h \]
Calculando o logaritmo dos dois lados e sabendo que logaritmo é uma função 
monotonicamente crescente temos que:
\begin{align*}
     \lg 2^h  &\geq  \lg \frac{n!}{n} \\
     h & \geq \lg n! - \lg n  \\
     h & \geq n\lg n - \lg n \\
\end{align*}
Note $n\lg n - \lg n = \Theta(n \lg n)$, portanto podemos concluir que para uma 
fração de $1/n$ das permutações também mantem-se gasto de tempo de $\Omega(n 
\lg 
n)$ no pior caso.
\\[6pt]
Trocando a "metade" pela fração de $1/2^n$, podemos utilizar o mesmo raciocínio 
para calcular o pior caso de qualquer algoritmo de ordenação baseada em 
comparações. Seja $l$ o número de folhas e $h$ a altura da árvore teremos a 
seguinte relação:
\[ \frac{n!}{2^n} \leq l \leq 2^h \]
Calculando o logaritmo dos dois lados e sabendo que logaritmo é uma função 
monotonicamente crescente temos que:
\begin{align*}
     \lg 2^h  &\geq  \lg (\frac{n!}{2^n}) \\
     h & \geq \lg (n!) - \lg 2^n \\
     h & \geq n \lg n - n \\
\end{align*}
Note que $n \lg n - n = \Theta(n \lg n)$, portanto da relação acima temos que 
mesmo para a fração de $1/2^n$ de permutações os algoritmos de ordenação que 
utilização comparações consomem tempo $\Omega(n \lg n)$ no pior caso.
\pagebreak